%%% Declaration of universal commands

\DeclareUnicodeCharacter{00B4}{`}

%%% References %%%
\newcommand{\figref}[1]{Fig.~\ref{#1}}              % Output: Fig. 1 (bei Subgrafik wird ausgegeben: Abb. 2a)
\newcommand{\subfigref}[1]{Fig.~\subref{#1}}        % Output: Fig. a (referenziert nur die Subgrafik, ohne Nennung der Supergrafik)
\newcommand{\tabref}[1]{Tab.~\ref{#1}}              % Output: Tab. 1
\newcommand{\chpref}[1]{chapter~\ref{#1}}           % Output: chapter 1
\newcommand{\secref}[1]{section~\ref{#1}}           % Output: section 1
\newcommand{\subsecref}[1]{section~\ref{#1}}        % Output: section 1
\newcommand{\listref}[1]{code~\ref{#1}}             % Output: code 1
\newcommand{\listlineref}[2]{code~\ref{#1}, #2}     % Output: code 1, Zeile 2 (wenn Argument 2 'Zeile 2' ist)
\renewcommand{\eqref}[1]{Eq.~\ref{#1}}              % Output: Eq. 1

%% Links
\newcommand{\mailto}[1]{\href{mailto:#1}{#1}}

% typografische Feinheiten
\newcommand{\eg}{\mbox{e.\,g.}\xspace}				% Output: e. g. (with small space and no hyphenation)
\newcommand{\page}[1]{\mbox{p.\,#1}}                % Output: p. 1
\newcommand{\pages}[1]{\mbox{pp.\,#1}}              % Output: pp. 1

\newcommand{\elq}{`}
\newcommand{\erq}{'}
\newcommand{\elqq}{``}
\newcommand{\erqq}{''}

\newcommand{\skipline}{\vspace{\baselineskip}} % leave a free line

% Smaller Minus Sign
\newcommand{\n}{\raisebox{.75pt}{-}}
\newcommand{\minus}{\!-\!}

% Mathematical spelling of decimal power , i. g.: 1E-2
\newcommand{\expneg}[2]{#1\mbox{\scshape{e}-}#2}    % Output: 1E-2 (with small capital E)
\newcommand{\exppos}[2]{#1\mbox{\scshape{e}}#2}     % Output: 1E2 (with small capital E)

% Symbols
% Nicer spelling of "C++"
\newcommand{\CC}{C\nolinebreak\hspace{-.05em}\raisebox{.4ex}{\tiny\bf +}\nolinebreak\hspace{-.10em}\raisebox{.4ex}{\tiny\bf +}}
