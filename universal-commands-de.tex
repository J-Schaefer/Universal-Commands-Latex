%%% Deklaration universeller Befehle

\DeclareUnicodeCharacter{00B4}{`}

%%% Neue Referenzen %%%
\newcommand{\figref}[1]{Abb.~\ref{#1}}                  % Ausgabe: Abb. 1 (bei Subgrafik wird ausgegeben: Abb. 2a)
\newcommand{\subfigref}[1]{Abb.~\subref{#1}}            % Ausgabe: Abb. a (referenziert nur die Subgrafik, ohne Nennung der Supergrafik)
\newcommand{\tabref}[1]{Tab.~\ref{#1}}                  % Ausgabe: Tab. 1
\newcommand{\chpref}[1]{Kapitel~\ref{#1}}               % Ausgabe: Kapitel 1
\newcommand{\secref}[1]{Abschnitt~\ref{#1}}             % Ausgabe: Abschnitt 1
\newcommand{\subsecref}[1]{Abschnitt~\ref{#1}}          % Ausgabe: Abschnitt 1
\newcommand{\listref}[1]{Quellcode~\ref{#1}}            % Ausgabe: Quellcode 1
\newcommand{\listlineref}[2]{Quellcode~\ref{#1}, #2}    % Ausgabe: Quellcode 1, Zeile 2 (wenn Argument 2 'Zeile 2' ist)
\renewcommand{\eqref}[1]{Formel~\ref{#1}}               % Ausgabe: Formel 1

%% Links
\newcommand{\mailto}[1]{\href{mailto:#1}{#1}}

% typografische Feinheiten
\newcommand{\zb}{\mbox{z.\,B.}\xspace}      % Ausgabe: z. B. (mit kleinem Abstand und keiner Silben-/Worttrennung)
\newcommand{\dahe}{\mbox{d.\,h.}\xspace}    % Ausgabe: d. h. (mit kleinem Abstand und keiner Silben-/Worttrennung)
\newcommand{\page}[1]{\mbox{S.\,#1}}

\newcommand{\elq}{`}
\newcommand{\erq}{'}
\newcommand{\elqq}{``}
\newcommand{\erqq}{''}

\newcommand{\skipline}{\vspace{\baselineskip}}  % leave a free line

% kleineres Minus-Zeichen
\newcommand{\n}{\raisebox{.75pt}{-}}
\newcommand{\bis}{\!-\!}

% Mathematische Angabe der Zehnerpotenz, z. B.: 1E-2
\newcommand{\expneg}[2]{#1\mbox{\scshape{e}-}#2}    % Ausgabe: 1E-2 (mit verkleinertem E)
\newcommand{\exppos}[2]{#1\mbox{\scshape{e}}#2}     % Ausgabe: 1E2 (mit verkleinertem E)

% Symbole
% Schönere Schreibweise des Symbols für "C++"
\newcommand{\CC}{C\nolinebreak\hspace{-.05em}\raisebox{.4ex}{\tiny\bf +}\nolinebreak\hspace{-.10em}\raisebox{.4ex}{\tiny\bf +}}
